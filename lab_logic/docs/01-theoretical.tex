\chapter{Теоретический раздел}

\section{Логика высказываний}

Повествовательное предложение, о котором можно сказать истинно оно или ложно, называется \textbf{высказываем}. Для высказываний вводятся буквы, которые называются атомами или пропозициональными переменными. Из простых высказываний с помощью логических связок строятся сложные высказывания.

\textbf{Общезначимая формула} --- формула, истинная при всех интерпретациях. Остальные формулы \textbf{выполнимы}, если они истины в какой-либо интерпретации.

Для решения практических задач необходимо приводить все формулы в КНФ. Алгоритм приведения формулы логики высказываний к КНФ:

\begin{enumerate}
    \item Устранить связки импликаций и эквиваленций используя соотношения:
    \begin{eqnarray*}
        A \rightarrow B = \neg A \vee B = \neg (A \& \neg B) \\
        A \leftrightarrow B = (A \rightarrow B) \& (B \rightarrow A)
    \end{eqnarray*}
    \item Продвинуть отрицания до атомов используя законы де Моргана:
    \begin{eqnarray*}
        \neg (A \& B) = \neg A \vee \neg B \\
        \neg (A \vee B) = \neg A \& \neg B
    \end{eqnarray*}
    \item Применить закон дистрибутивности:
    \begin{eqnarray*}
        A \vee (B \& C) = (A \vee B) \& (A \vee C)
    \end{eqnarray*}
\end{enumerate}

Формула $F$ является \textbf{логическим следствием} $n$ формул $A_i$, если она истина во всех интерпретациях, в которых истины все $A_i$. Из этого вытекает, что импликация конъюнкций $A_i$ к $F$ общезначима:

\begin{equation*}
    (A_1 \& A_2 \& \dots \& A_n) \rightarrow F
\end{equation*}

Общезначимость не доказывается, так как это требует рассмотрения всех интерпретаций. Поэтому переходят к доказательству противоречивости отрицания $\neg F$. Формула представляется как конъюнкция дизъюнктов. Если хоть один дизъюнкт будет ложным --- то мы доказали противоречие. В основе доказательства лежит процедура резолюции, которая вводит понятие резольвенты двух дизъюнктов. Если у нас есть два дизъюнкта, в которые входит один и тот же атом, но с разными знаками (эти атомы называются контрарной парой в дизъюнктах), то можем получить дизъюнкт, который является объединением атомов первого и второго дизъюнкта за исключением контрарной пары. Этот дизъюнкт называется резольвентой и легко показать, что он является логическим следствием двух дизъюнктов.


Алгоритм вывода по методу резолюции:

\begin{enumerate}
    \item Принять отрицание заключения.
    \item Привести все формулы посылок (аксиом) и отрицание цели к КНФ.
    \item Выделить список дизъюнктов.
    \item Если существует пара дизъюнктов, содержащих контрарные атомы, то эти дизъюнкты объединяются с удалением контрарной пары и получаем новый дизъюнкт --- резольвенту, которая является логическим следствием и добавляется к исходному множеству дизъюнктов.
\end{enumerate}

Исходный алгоритм анализа дизъюнктов --- полный перебор. Если на каком-то шаге мы получим два дизъюнкта по типу $A, \neg A$, резольвента будет пустым дизъюнктом или ложью. Если не получили пустого дизъюнкта --- продолжаем перебор на новом множестве.

Нулевой пример:

$$\frac{A \vee B, ~~ (A \rightarrow C), ~~ B \rightarrow D}{C \vee D}$$

\begin{enumerate}
    \item $\neg (C \vee D) = (\neg C \wedge \neg D)$
    \item $(A \rightarrow C) = (\neg A \vee C)$
    \item $(B \rightarrow D) = (\neg B \vee D)$
    \item $K = (A \vee B, \neg A \vee C, \neg B \vee D, \neg C, \neg D)$
    \item $[A \vee C + \neg C]: K = (A \vee B, \neg A, \neg B \vee D, \neg D)$
    \item $[A \vee B + \neg A]: K = (B, \neg B \vee D, \neg D)$
    \item $[B + \neg B \vee D]: K = (D, \neg D)$
    \item $K = ()$
\end{enumerate}


Реализовали стратегию унитарности, когда искали дизъюнкт с наименьшим количеством атомов. В общем случае построение резольвент выполняется полным перебором для текущего множества дизъюнктов, однако существуют модификации этого алгоритма:

\begin{enumerate}
    \item \textbf{Стратегия унитарности} --- выбирать самый короткий дизъюнкт.
    \item \textbf{Линейная резолюция} --- задаем исходное множество дизъюнктов, получаем резольвенту для пары дизъюнктов из него. Один из дизъюнктов целесообразно взять из отрицания цели. На следующем шаге искать резольвенту, где один дизъюнкт --- последняя резольвента, второй дизъюнкт из исходного множества. Либо один дизъюнкт должен быть получен как резольвента.
    \item \textbf{Идея опорного множетсва}. Исходное множество дизъюнктов разделяют на множество (исходное), которое относится к аксиомам, и множество (опорное), полученное из отрицания цели. Резольвенты формируются один дизъюнкт из опорного множетсва, другой из исходного и полученных резольвент.
\end{enumerate}

\section{Логика предикатов первого порядка}

Логика предикатов поволяет описывать как свойства, так и отношения между объектами.
Под \textbf{предикатом} будем принимать повествовательное предложение, которое содержит переменные и при означивании переменных константами становится высказыванием, о котором можно сказать истинно оно или ложно.

\textbf{Предикат} на некотором множетсве $W$ есть логическая функция, которая при означивании аргументов превращается в высказывание со значениями истина или ложь. Эта логическая функция может зависеть от одной переменной, от двух или от $n$ переменных. В случае одной переменной --- это отражение свойства объекта. В случае $n$ переменных --- это отражение связи между $n$ объектами. Поэтому говоря о предикатах, необходимо задать интерпретацию --- множество объектов, на которых предикат определен. В случае предиката от одной переменной, переменная принимает значения объекта из фиксированной области.

Также вводятся функци от n объектов, котороые отображают n объектов в 1 объект из наших областей. Константа и переменная называются термом. Функция от термов --- тоже терм. $f(t_1, t_2, t_3)$ --- элементарная формула.

Следующий тип формул --- использование кванторов всеобщности и существования. Роли переменных связывают их с предметной областью. Формулы логики предикатов: Квантор общности --- для предиката с одной переменной означает, что он истиннен для всей области определения. Множество истинности предиката --- те значения переменных, на которых предикат становится истинным. Квантор существования означает, что есть хотя бы один объект, для которого это свойство истинно. Если мы имеем предикат от 1 переменной, то накладывая квантор на него мы получаем высказывание.

Рассмотрим $n$-местные предикаты. Часть переменных может находиться под кванторами. Эти переменные называются связанными. Остальные --- свободными. А арность нового предиката становится уже меньше.

Правила проноса отрицания для кванторов:
\begin{eqnarray*}
    \neg \forall x P(x) = \exists x \neg P(x) \\
    \neg \exists x P(x) = \forall x \neg P(x)
\end{eqnarray*}
