\documentclass{bmstu}

\bibliography{biblio}
\usepackage{enumitem}
\usepackage{amsmath,amssymb,amsthm}

\usepackage{algorithm}
\usepackage{algpseudocode}
\floatname{algorithm}{Алгоритм}
\captionsetup[ruled]{labelsep=period}
\makeatletter
\@addtoreset{algorithm}{chapter}% algorithm counter resets every chapter
\makeatother
\renewcommand{\thealgorithm}{\thechapter.\arabic{algorithm}}%

%Перевод команд псевдокода
\algrenewcommand\algorithmicwhile{\textbf{До тех пока}}
  \algrenewcommand\algorithmicdo{\textbf{выполнять}}
  \algrenewcommand\algorithmicrepeat{\textbf{Повторять}}
  \algrenewcommand\algorithmicuntil{\textbf{Пока выполняется}}
  \algrenewcommand\algorithmicend{\textbf{Конец}}
  \algrenewcommand\algorithmicif{\textbf{Если}}
  \algrenewcommand\algorithmicelse{\textbf{иначе}}
  \algrenewcommand\algorithmicthen{\textbf{тогда}}
  \algrenewcommand\algorithmicfor{\textbf{Цикл}}
  \algrenewcommand\algorithmicforall{\textbf{Выполнить для всех}}
  \algrenewcommand\algorithmicfunction{\textbf{Функция}}
  \algrenewcommand\algorithmicprocedure{\textbf{Процедура}}
  \algrenewcommand\algorithmicloop{\textbf{Зациклить}}
  \algrenewcommand\algorithmicrequire{\textbf{Условия:}}
  \algrenewcommand\algorithmicensure{\textbf{Обеспечивающие условия:}}
  \algrenewcommand\algorithmicreturn{\textbf{Вернуть}}
  \algrenewtext{EndWhile}{\textbf{Конец цикла}}
  \algrenewtext{EndLoop}{\textbf{Конец зацикливания}}
  \algrenewtext{EndFor}{\textbf{Конец цикла}}
  \algrenewtext{EndFunction}{\textbf{Конец функции}}
  \algrenewtext{EndProcedure}{\textbf{Конец процедуры}}
  \algrenewtext{EndIf}{\textbf{Конец условия}}
  \algrenewtext{EndFor}{\textbf{Конец цикла}}
  \algrenewtext{BeginAlgorithm}{\textbf{Начало алгоритма}}
  \algrenewtext{EndAlgorithm}{\textbf{Конец алгоритма}}
  \algrenewtext{BeginBlock}{\textbf{Начало блока. }}
  \algrenewtext{EndBlock}{\textbf{Конец блока}}
  \algrenewtext{ElseIf}{\textbf{иначе если }}

\begin{document}

\makereporttitle
    {Информатика, искусственный интеллект и системы управления} % Название факультета
    {Программное обеспечение ЭВМ и информационные технологии} % Название кафедры
    {лабораторной работе №~6} % Название работы (в дат. падеже)
    {Экспертные системы} % Название курса (необязательный аргумент)
    {Обратный дедуктивный вывод в обобщенных правилах продукции} % Тема работы
    {} % Номер варианта (необязательный аргумент)
    {Клименко~А.~К./ИУ7-31М} % Номер группы/ФИО студента (если авторов несколько, их необходимо разделить запятой)
    {Русакова~З.~Н.} % ФИО преподавателя

\chapter*{Введение}

Цель лабораторной работы --- приобретение практических навыков реализации алгоритмов унификации и резолюции для произвольного множества дизъюнктов в логике предикатов первого порядка.

Задачи работы:
\begin{enumerate}
    \item Формализовать грамматику выражений алгебры логики предикатов первого порядка для обработки в автоматическом режиме.
    \item Разработать программу, реализующую метод резолюции для произвольных выражений на языке алгебры логики предикатов.
    \item Протестировать программу для различных теорем и логических задач.
\end{enumerate}


\chapter{Теоретический раздел}

\section{Обобщенное правило продукции}

Пусть $x$ --- задан на конкретном множестве, а $c$ --- объект этого множества, тогда
$$\frac{A(c), A(x) \rightarrow B(x)}{B(c)}.$$

В общем случае, если в посылке конъюнкция и заданы подстановки для каждого $P_i$, то мы можем получить композицию подстановок.

Отсутствие направления вывода в методе резолюции называется \textbf{потерей импликативности}.
Импликативный вывод используется для простых случаев.

\section{Система дедукции на основе обобщенных правил продукции}

Определенное выражение является либо атомарным (атомом), либо представляет собой импликацию антицидентом (предпосылкой) которой является конъюнкция положительных атомов, а консеквент --- единственный положительный атом.

В определенных выражениях в базе знаний не используются функциональные символы, но определенные выражения содержат направление вывода, что намного повышает эффективность систем вывода и упрощает его. Метод резолюции позволяет обработать более сложные представления знаний, что определенные выражения не всегда позволяют, но процедура резолюции влечет потерю импликативности, то есть нет направления вывода.

Пример:
\begin{eqnarray*}
  (\neg A \& \neg B) &\rightarrow& C, \\
  (\neg B \& \neg C) &\rightarrow& A
\end{eqnarray*}
дают одинаковые дизъюнкты, но видим, что в исходных формулах разное направление вывода.

\section{Алгоритм обратного дедуктивного вывода}

Обратная дедукция для базы знаний из определенных выражений производится методом поиска в глубину. Структуры данных: переменные, константы, атом. База правил представлена списком правил. Правило содержит список входных атомов, которые соответствуют входным вершинам, один выходной атом (выходная вершина), номер правила, флаг доказано/недоказано и метку (выбрано или нет). В атоме тоже вводится флаг (помимо списка переменных) доказан атом или не доказан. Факты --- это атомы, в которых стоят константы. Они соответствуют закрытым вершинам в графах И-ИЛИ (мы их задаем) и еще есть целевой атом (целевая вершина). Возможно что целевой атом имеет константы, а может не имеет.

Целевой атом записываем в голову стека. Факты записываем в список закрытых вершин. В процессе поиска формируем стек открытых вершин (атомов); список подстановок для текущего шага; список закрытых атомов / закрытых вершин, которые добавляются к исходному списку фактов; список закрытых правил, которые содержат дерево решения.

Метод поиска остается без изменений. Пока оба флага выставлены в единицу, вызываем метод потомков на каждом шаге поиска. Изменяется метод потомков и метод разметки. В этих методах необходимо проводить унификацию и формировать подстановки. В методе потомков выполняется унификация атома подцели (из стека) и выходной атом рассматриваемого правила. Унифицируются подцель и выходной атом правила. Если унификация успешна, то формируем подстановки. Причем, константа может стоять как в подцели так и в выходном атоме правила. Если выходной атом получает константу, то её необходимо распространить на все атомы рассматриваемого правила, в которое входит переменная, получившая константу. Если константу получают переменные подцели, то она ставится в подцель. При этом можно рассмотреть 2 пути: если подцель получила константу, то распространить эту константу на оставшиеся в стеке атомы последнего правила. Альтернатива --- можно не распространять константу из подцели, а оставить их с переменными, --- они получат значения потом при их доказательстве.

Итог: выбираем подцель, находим первое правило, у которого подцель унифицируется с выходной вершиной. Нам нужно проверить, сколько атомов у этого правила уже закрыто (входные вершины доказаны). Для этого мы каждый атом входной проверяем, не находится ли он в базе. Если он есть, то его переменные получают значения. Эти значения распространяем в другие атомы. В стек записываем только те атомы, которые не закрываются фактами. Если мы записали подцели, то номер правила помещаем в список открытых правил.

Подстановка необходима на одном шаге раскрытия подцели и формирования новых подцелей. Потом она уже не нужна, но можно использовать.

Следующий шаг --- если у найденного правила все атомы доказаны. Это правило добавляем в список закрытых и начинаем разметку. В разметке мы должны выходную вершину правила добавить к фактам. В результате подстановки, мы должны полученные значения переменных предыдущих правил распространить на эти же переменные в оставшихся недоказанных атомах (если они есть). Идет обратное распространение переменных (снизу вверх в подцели). Соответственно, из головы стека подцель удаляем. Если недоказанные атомы для последнего правила были, то вызываем потомков. Если недоказанных атомов больше нет, то продолжаем цикл разметки.

Рассмотрим построение для наших правил.

Факты:
\begin{enumerate}
  \item $O(N,M1)$
  \item $M(M1)$
  \item $A(W)$
  \item $E(N,A1)$
\end{enumerate}

Правила:
\begin{enumerate}
  \item $A(x) \& W(y) \& S(x, y, z) \& H(z) \rightarrow C(x)$
  \item $M(x_2) \& O(N, x_2) \rightarrow S(W, x_2, N)$
  \item $M(x_1) \rightarrow W(x_1)$
  \item $E(x_3, A1) \rightarrow H(x_3)$
\end{enumerate}

Цель: $C(x_0)$

Шаги:
\begin{enumerate}
  \item стек: $[ C(x_0) ]$

  $C(x_0)$ унифицируется с выходным атомом правила 1. Подстановка: $\{x_0/x\}$
  Мы должны проверить, какие атомы этого правила закрыты. Берем $A(x)$ ищем в базе фактов, находим $A(W)$ --- $x$ заменяется на константу $W$. Распространяем переменную на остальные факты. $A(W)$ в стек не пишем, рассматриваем следующую подцель. $W(y)$ в базе фактов нет, поэтому кладем в стек:
	
  стек: $[ W(y), S(W,y,z), H(z), C(W) ]$

  \item второй шаг метода потомков --- $W(y)$ унифицируется с выходом правила 3. Подстановка: $\{y/x_1\}$. Нам нужно проверить у этого правила, какие вершины не доказаны и что писать в стек, в нашем случае при доказательстве он сопоставляется с фактом $M(M1)$, когда мы будем доказывать  входную вершину получим подстановку $\{M1/y\}$. Правило 3 закрываем. Добавляем факт $W(M1)$ в базу фактов. Распространяем подстановку $\{M1/y\}$ на оставшиеся атомы.

	стек: $[ S(W, M1, z), H(z), C(W) ]$, а в базе фактов новый факт: $W(M1)$. \\
	Из разметки вышли --- $W(y)$ убираем, а недоказанные подцели есть. Теперь будем доказывать $S(W, M1, z)$.

  \item В результате унификации $S(W,M1,z)$ с выходом правила $S(W, x_2, N)$ надо распространять в оба направления --- $x_2$ вниз (в подцели для $S$), $z$ вправо (в $H(z)$). Подцели $S$ будут доказаны из фактов, из этого следует, что вершина $S(W, M1, N)$ --- новый факт, который мы добавим в базу фактов и уберем из стека.

  \item Подцель $H(N)$. Правило 4. Входной модуль $E(x_3, A1)$. В результате распространения получаем $E(N, A1)$. Такой факт у нас есть. $H(N)$ в новые факты, убираем из стека.
  
  \item Вызываем разметку для $C(W)$ --- удаляем из стека. Стек пуст --- исходное утверждение доказано.
\end{enumerate}

\chapter{Практический раздел}

\section{Базовые структуры вершины и правила}

\begin{lstlisting}
struct Node {
    int number;
    bool forbidden = false;
    bool closed = false;
};

struct Rule {
    std::vector<int> srcNodes;
    int dstNode;
    int number;
    bool visited = false;
    bool forbidden = false;
    unsigned int openIndex = 0;
};
\end{lstlisting}

\section{Класс поиска в графе}

\begin{lstlisting}
class GraphSearch {
public:
    GraphSearch(std::list<Rule> rules, std::vector<int> srcNode, int dstNode);
    std::list<int> DoDepthFirstSearch();

private:
    int DescendantsDFS(int node);
    void Backtrack(int node);
    void Mark(int node);

    std::list<Rule> m_rules;
    std::map<int, Node> m_nodes;
    std::map<int, Rule *> m_rulesRef;
    std::list<int> m_openNodes;
    std::list<int> m_openRules;
    std::list<int> m_closedNodes;
    std::list<int> m_closedRules;
    std::vector<int> m_srcNodes;
    bool m_foundSolution = false;
    bool m_noSolution = false;
};
\end{lstlisting}

\section{Реализация методов класса поиска}

\begin{lstlisting}
GraphSearch::GraphSearch(std::list<Rule> rules, std::vector<int> srcNodes, int dstNode) : m_rules(std::move(rules)), m_srcNodes(std::move(srcNodes)) {
  if (m_srcNodes.size() == 0)
    m_noSolution = true;
  else if (std::find(m_srcNodes.begin(), m_srcNodes.end(), dstNode) != m_srcNodes.end())
    m_foundSolution = true;
  else {
    for (auto node : m_srcNodes) {
      m_nodes[node].closed = true;
      m_closedNodes.push_back(node);
    }
    for (auto &rule : m_rules)
      m_rulesRef[rule.number] = &rule;
    m_openNodes.push_back(dstNode);
  }
}

std::list<int> GraphSearch::DoDepthFirstSearch() {
  while (!m_foundSolution && !m_noSolution) {
    int node = m_openNodes.back();
    int count = DescendantsDFS(node);
    if (m_foundSolution) {
      Mark(node);
      if (m_foundSolution) break;
    } else if (count == 0 && !m_openNodes.empty()) {
      Backtrack(node);
    } else if (m_openNodes.empty()) {
      m_noSolution = true;
      break;
    }
  }
  return m_noSolution ? {} : m_closedRules;
}

int GraphSearch::DescendantsDFS(int node) {
  int count = 0;
  for (auto &rule : m_rules) {
    if (rule.visited || rule.dstNode != node || rule.forbidden)
      continue;
    bool all_closed = true;
    for (auto srcNode : rule.srcNodes) {
      const Node &nodeRef = m_nodes[srcNode];
      if (nodeRef.forbidden) {
        rule.forbidden = true;
        break;
      }
      all_closed = all_closed && nodeRef.closed;
    }
    if (rule.forbidden) continue;
    m_openRules.push_back(rule.number);
    if (all_closed) {
      m_foundSolution = true;
      break;
    }
    m_rulesRef[rule.number]->openIndex = m_openNodes.size();
    for (auto srcNode : rule.srcNodes)
      if (!m_nodes[srcNode].closed)
        m_openNodes.push_back(srcNode);
    rule.visited = true;
    ++count;
  }
  return count;
}

void GraphSearch::Backtrack(int node) {
  while (true) {
    // node at top of open node stack needs to be forbidden
    m_nodes[node].forbidden = true;
    // then rule at top of open rules stack needs to be forbidden
    if (m_openRules.empty()) {
      m_noSolution = true;
      break;
    }
    int ruleNum = m_openRules.back();
    Rule *ruleRef = m_rulesRef[ruleNum];
    int dstNode = ruleRef->dstNode;
    ruleRef->forbidden = true;
    //  all input nodes of that rule must be removed from open nodes
    for (const auto &rule : m_rules) {
      if (rule.number == ruleNum) {
        while (ruleRef->openIndex < m_openNodes.size())
          m_openNodes.pop_back();
        break;
      }
    }
    m_openRules.pop_back();
    // new open rule must be searched to close previous node in open stack
    // 1. check if previous open rule resolves current top of open nodes
    // 2. if true - continue with this rule
    //    if not - remove this rule and corresponding nodes and retry
    if (m_openRules.empty())
      m_noSolution = true;
    else if (m_rulesRef[m_openRules.back()]->dstNode == dstNode)
      break;
    node = dstNode;
  }
}

void GraphSearch::Mark(int node) {
  // 0. mark given node as closed
  // 1. pop all rules that have same dst node
  // 2. if current top of opened rules has closed inputs - repeat
  // 3. dont forget to update m_foundSolution variable if real solution was found!
  while (m_foundSolution) {
    m_nodes[node].closed = true;
    m_closedNodes.push_back(node);
    m_closedRules.push_back(m_openRules.back());
    while (!m_openRules.empty() && m_rulesRef[m_openRules.back()]->dstNode == node)
      m_openRules.pop_back();
    // real solution found
    if (m_openRules.empty())
      return;
    // remove all open nodes up until given node
    while (m_openNodes.back() != node)
      m_openNodes.pop_back();
    m_openNodes.pop_back();
    // check current top
    int rule = m_openRules.back();
    bool all_closed = true;
    for (auto srcNode : m_rulesRef[rule]->srcNodes) {
      if (!m_nodes[srcNode].closed) {
        all_closed = false;
        break;
      }
    }
    if (all_closed)
      node = m_rulesRef[rule]->dstNode;
    else m_foundSolution = false;
  }
}
\end{lstlisting}

\clearpage

\section{Тестирование ПО}

Для тестирования реализации алгоритма поиска была составлена база знаний, соответствующая графу, представленному на рисунке \ref{fig:graph0}.

\begin{figure}[h!]
    \centering
    \includegraphics[width=\linewidth]{img/kb.pdf}
    \caption{Граф составленной базы знаний}
    \label{fig:graph0}
\end{figure}

\begin{table}[h!]
    \centering
    \caption{Тестовые данные для проверки реализации алгоритма}
    \begin{tabular}{|c|c|c|c|}
        \hline
        Входные & Выходная & Ожидаемый & Фактический \\
        вершины & вершина & результат & результат \\
        \hline
        \hline
        $1,2,4,8,9$ & $5$ & $100,101$ & $100,101$ \\
		$2,3,4,12,13,21,22,23,24$ & $11$ & $111,105,109$ & $111,105,109$ \\
		$8,9,6,7,12$ & $10$ & $\O$ & $\O$ \\
		$2,3,4,12,14,15,23,24$ & $11$ & $111,106,105,109$ & $111,106,105,109$ \\
        \hline
    \end{tabular}
\end{table}

\chapter*{Выводы}

В ходе лабораторной работы была разработана программа, осуществляющая обратный поиск в глубину в графе И-ИЛИ.


\end{document}