
\chapter{Теоретический раздел}

\section{Обобщенное правило продукции}

Пусть $x$ --- задан на конкретном множестве, а $c$ --- объект этого множества, тогда
$$\frac{A(c), A(x) \rightarrow B(x)}{B(c)}.$$

В общем случае, если в посылке конъюнкция и заданы подстановки для каждого $P_i$, то мы можем получить композицию подстановок.

Отсутствие направления вывода в методе резолюции называется \textbf{потерей импликативности}.
Импликативный вывод используется для простых случаев.

\section{Система дедукции на основе обобщенных правил продукции}

Определенное выражение является либо атомарным (атомом), либо представляет собой импликацию антицидентом (предпосылкой) которой является конъюнкция положительных атомов, а консеквент --- единственный положительный атом.

В определенных выражениях в базе знаний не используются функциональные символы, но определенные выражения содержат направление вывода, что намного повышает эффективность систем вывода и упрощает его. Метод резолюции позволяет обработать более сложные представления знаний, что определенные выражения не всегда позволяют, но процедура резолюции влечет потерю импликативности, то есть нет направления вывода.

Пример:
\begin{eqnarray*}
  (\neg A \& \neg B) &\rightarrow& C, \\
  (\neg B \& \neg C) &\rightarrow& A
\end{eqnarray*}
дают одинаковые дизъюнкты, но видим, что в исходных формулах разное направление вывода.

\section{Алгоритм обратного дедуктивного вывода}

Обратная дедукция для базы знаний из определенных выражений производится методом поиска в глубину. Структуры данных: переменные, константы, атом. База правил представлена списком правил. Правило содержит список входных атомов, которые соответствуют входным вершинам, один выходной атом (выходная вершина), номер правила, флаг доказано/недоказано и метку (выбрано или нет). В атоме тоже вводится флаг (помимо списка переменных) доказан атом или не доказан. Факты --- это атомы, в которых стоят константы. Они соответствуют закрытым вершинам в графах И-ИЛИ (мы их задаем) и еще есть целевой атом (целевая вершина). Возможно что целевой атом имеет константы, а может не имеет.

Целевой атом записываем в голову стека. Факты записываем в список закрытых вершин. В процессе поиска формируем стек открытых вершин (атомов); список подстановок для текущего шага; список закрытых атомов / закрытых вершин, которые добавляются к исходному списку фактов; список закрытых правил, которые содержат дерево решения.

Метод поиска остается без изменений. Пока оба флага выставлены в единицу, вызываем метод потомков на каждом шаге поиска. Изменяется метод потомков и метод разметки. В этих методах необходимо проводить унификацию и формировать подстановки. В методе потомков выполняется унификация атома подцели (из стека) и выходной атом рассматриваемого правила. Унифицируются подцель и выходной атом правила. Если унификация успешна, то формируем подстановки. Причем, константа может стоять как в подцели так и в выходном атоме правила. Если выходной атом получает константу, то её необходимо распространить на все атомы рассматриваемого правила, в которое входит переменная, получившая константу. Если константу получают переменные подцели, то она ставится в подцель. При этом можно рассмотреть 2 пути: если подцель получила константу, то распространить эту константу на оставшиеся в стеке атомы последнего правила. Альтернатива --- можно не распространять константу из подцели, а оставить их с переменными, --- они получат значения потом при их доказательстве.

Итог: выбираем подцель, находим первое правило, у которого подцель унифицируется с выходной вершиной. Нам нужно проверить, сколько атомов у этого правила уже закрыто (входные вершины доказаны). Для этого мы каждый атом входной проверяем, не находится ли он в базе. Если он есть, то его переменные получают значения. Эти значения распространяем в другие атомы. В стек записываем только те атомы, которые не закрываются фактами. Если мы записали подцели, то номер правила помещаем в список открытых правил.

Подстановка необходима на одном шаге раскрытия подцели и формирования новых подцелей. Потом она уже не нужна, но можно использовать.

Следующий шаг --- если у найденного правила все атомы доказаны. Это правило добавляем в список закрытых и начинаем разметку. В разметке мы должны выходную вершину правила добавить к фактам. В результате подстановки, мы должны полученные значения переменных предыдущих правил распространить на эти же переменные в оставшихся недоказанных атомах (если они есть). Идет обратное распространение переменных (снизу вверх в подцели). Соответственно, из головы стека подцель удаляем. Если недоказанные атомы для последнего правила были, то вызываем потомков. Если недоказанных атомов больше нет, то продолжаем цикл разметки.

Рассмотрим построение для наших правил.

Факты:
\begin{enumerate}
  \item $O(N,M1)$
  \item $M(M1)$
  \item $A(W)$
  \item $E(N,A1)$
\end{enumerate}

Правила:
\begin{enumerate}
  \item $A(x) \& W(y) \& S(x, y, z) \& H(z) \rightarrow C(x)$
  \item $M(x_2) \& O(N, x_2) \rightarrow S(W, x_2, N)$
  \item $M(x_1) \rightarrow W(x_1)$
  \item $E(x_3, A1) \rightarrow H(x_3)$
\end{enumerate}

Цель: $C(x_0)$

Шаги:
\begin{enumerate}
  \item стек: $[ C(x_0) ]$

  $C(x_0)$ унифицируется с выходным атомом правила 1. Подстановка: $\{x_0/x\}$
  Мы должны проверить, какие атомы этого правила закрыты. Берем $A(x)$ ищем в базе фактов, находим $A(W)$ --- $x$ заменяется на константу $W$. Распространяем переменную на остальные факты. $A(W)$ в стек не пишем, рассматриваем следующую подцель. $W(y)$ в базе фактов нет, поэтому кладем в стек:
	
  стек: $[ W(y), S(W,y,z), H(z), C(W) ]$

  \item второй шаг метода потомков --- $W(y)$ унифицируется с выходом правила 3. Подстановка: $\{y/x_1\}$. Нам нужно проверить у этого правила, какие вершины не доказаны и что писать в стек, в нашем случае при доказательстве он сопоставляется с фактом $M(M1)$, когда мы будем доказывать  входную вершину получим подстановку $\{M1/y\}$. Правило 3 закрываем. Добавляем факт $W(M1)$ в базу фактов. Распространяем подстановку $\{M1/y\}$ на оставшиеся атомы.

	стек: $[ S(W, M1, z), H(z), C(W) ]$, а в базе фактов новый факт: $W(M1)$. \\
	Из разметки вышли --- $W(y)$ убираем, а недоказанные подцели есть. Теперь будем доказывать $S(W, M1, z)$.

  \item В результате унификации $S(W,M1,z)$ с выходом правила $S(W, x_2, N)$ надо распространять в оба направления --- $x_2$ вниз (в подцели для $S$), $z$ вправо (в $H(z)$). Подцели $S$ будут доказаны из фактов, из этого следует, что вершина $S(W, M1, N)$ --- новый факт, который мы добавим в базу фактов и уберем из стека.

  \item Подцель $H(N)$. Правило 4. Входной модуль $E(x_3, A1)$. В результате распространения получаем $E(N, A1)$. Такой факт у нас есть. $H(N)$ в новые факты, убираем из стека.
  
  \item Вызываем разметку для $C(W)$ --- удаляем из стека. Стек пуст --- исходное утверждение доказано.
\end{enumerate}
